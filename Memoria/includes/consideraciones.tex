\mysection{Consideraciones técnicas} \label{sec:consideraciones}

    La confección de esta memoria se ha realizado en el entorno \textit{Overleaf} \cite{OVERLEAF}, un editor de texto \TeX\  \textit{online}. En particular, la versión de \TeX\  utilizada para la compilación del código fuente (accesible en el repositorio del trabajo \cite{TFGTEX}) es \texttt{LuaTeX}, una versión extendida de \texttt{pdfTeX} que utiliza \texttt{Lua} como lenguaje de \textit{scripting} embebido \cite{LUATEX}.
    \\ \\
    Esta memoria utiliza dos tipografías de manera adicional; \texttt{Roboto} \cite{ROBOTO} y \texttt{FiraCode} \cite{FIRACODE}. Sus licencias son \cite{LICAPACHE} y \cite{LICSIL}, respectivamente. Ambas son licencias permisivas que requieren su uso bajo atribución y mismas condiciones.
    \\ \\
    El sistema \textit{software} desarrollado, tal y como se describe en la sección de \hyperref[subsubsec:librerias]{librerías utilizadas}, aprovecha las funcionalidades de librerías cuyas licencias (a saber, MIT \cite{LICMIT}, Apache v2\cite{LICAPACHE} y BSD 3-Clause \cite{LICBSD3}) son también permisivas y su uso en este trabajo queda supeditado a la atribución de todas ellas. Por lo tanto, el sistema se ha publicado bajo la licencia Apache v2, con el objetivo de permitir la reutilización del \textit{software} bajo los mismos términos.
    \\ \\
    El sistema de citaciones utilizado en la \hyperref[sec:referencias]{bibliografía} es el sistema de referencias de \texttt{IEEE} \cite{IEEEREF}.

\newpage