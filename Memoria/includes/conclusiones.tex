\mysection{Conclusiones}

    Resumidamente, este trabajo ha buscado implementar una serie de funcionalidades que mejorasen la calidad y la accesibilidad de los datos en el campo de la contratación pública; desde el primer análisis en el que se evidenció la necesidad de realizar la portabilidad de los datos entre los esquemas \texttt{CODICE} y \texttt{OCDS}, hasta la propia publicación de los datos tras el despliegue del \textit{software} desarrollado.
    \\ \\
    En general, todos los objetivos planteados se han cumplido, que si bien no eran muchos, eran concretos y necesarios. En particular, el desarrollo realizado de un sistema capaz de admitir distintos tipos de modos de operación para múltiples casos de uso, y cuyo despliegue es ya operativo, es el mayor logro de este trabajo. 
    \\ \\
    Se anima a cualquier parte interesada de la comunidad de datos abiertos ha reutilizar dicho \textit{software} (información relativa a la licencia sobre la que se distribuye el \textit{software} en \hyperref[sec:consideraciones]{la última sección del trabajo}). Asimismo, se invita a cualquier desarrollador a realizar cualquier tipo de modificaciones que crea de interés mediante la creación de \textit{pull requests}, la metodología estándar de creación y cambio de funcionalidades en los proyectos alojados en repositorios de herramientas de integración continua como \textit{GitHub}.
    \\ \\
    Con respecto a las líneas futuras sobre las que se podría trabajar en el proyecto, quedan enumeradas a continuación las que se plantean más interesantes:
    \begin{itemize}
        \item Alcanzar una mayor completitud de la cantidad de reglas analizadas, descritas, implementadas y documentadas que el \textit{software} es capaz de procesar: en el momento de la publicación de este trabajo, la cantidad de elementos de \texttt{CODICE} con correspondencia a elementos \texttt{OCDS} es del 36\%, un porcentaje bajo pero que recoge la mayoría de los elementos de interés.
        \\
        \item Realizar el despliegue de una \textit{API} por encima de los datos abiertos publicados que sea capaz de explotarlos: ahora mismo, los datos enlazados son públicos y su consulta puede ser realizada mediante \textit{queries} \texttt{SPARQL} como las descritas en el \hyperref[annex:sparql]{anexo III}. Sin embargo, la construcción de una capa capaz de abstraer la complejidad de dichas \textit{queries} puede ser de interés.
        \\
        \item Integrar la ontología extendida en las ya existentes: con motivo de la utilización del sistema de lotes en \texttt{OCDS} mediante la extensión de mismo nombre, fue necesario crear una pequeña ampliación de la ontología \cite{MYONT} que fuera capaz de recoger las relaciones y entidades derivadas de la utilización de lotes.
        \\
        \item Pese a que en \hyperref[subsec:validacion]{una anterior sección} se ha probado la validación de los datos \texttt{OCDS}, tal y como se ha explicado en \hyperref[subsubsec:expediente]{la subsección de identificadores}, el sistema utiliza actualmente un prefijo \texttt{OCDS} ajeno (\texttt{ocds-1xraxc}), propio de los sistemas del Ayuntamiento de Zaragoza, debido a que \texttt{OCDS} sólo proporciona prefijos a entidades de mayor interés que un Trabajo de Fin de Grado. Por ello, realizar la solicitud de un prefijo para su uso en el presente sistema desarrollado debiera de ser uno de los próximos pasos.
        \\
        \item Construir herramientas de Inteligencia Artficial capaces de detectar a través de los datos posibles casos de delitos fiscales: ésta es quizá la línea de exploración futura más ambiciosa de todas, pero también la que mayores beneficios pueda tener. Debido a que la totalidad de contrataciones públicas que se licitan en España están obligadas a ser publicadas en los portales gubernamentales de los que el sistema desarrollado extrae los datos, todas las irregularidades también quedan de una manera u otra reflejadas. El desarrollo de herramientas que mediante \textit{IA} sean capaces de detectar aquellos patrones o anomalías que puedan evidenciar delitos fiscales como el fraude o la malversación de fondos públicos sería una línea a explorar de muchísimo interés.
    \end{itemize}
\newpage