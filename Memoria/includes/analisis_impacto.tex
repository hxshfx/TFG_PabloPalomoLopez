\mysection{Análisis del impacto del trabajo}

    Según lo previamente descrito y los resultados provistos por este trabajo, se pretende que éste tenga una fuerte componente práctica en el ámbito de la información de las licitaciones públicas. Por ello, se espera que el impacto de este trabajo sea significativo en este campo, el de los datos abiertos, que además tiene como algunas de sus bases fundamentales la cooperación e interoperabilidad de sistemas con el fin de proveer servicios que puedan ser accedidos y aprovechados por todos.
    \\ \\
    Si bien es cierto que el funcionamiento del sistema desarrollado y desplegado en este trabajo se basa fundamentalmente en la transformación de datos, se espera que gracias a esta transformación y a su publicación como datos enlazados se pueda construir y mejorar la base de conocimiento existente acerca del ámbito de las licitaciones de tal manera que sirva tanto de apoyo en la toma de decisiones de las entidades públicas, como de disuasión de algunos delitos fiscales que puedan ser encontrados mediante la explotación de los nuevos datos.
    \\ \\
    En relación con los Objetivos de Desarrollo Sostenible de la Agenda 2030 de la Organización de las Naciones Unidas \cite{ODS}, este trabajo se alinea con el objetivo decimosexto, \texttt{ODS 16} - \textit{Promover sociedades pacíficas e inclusivas para el desarrollo sostenible, facilitar el acceso a la justicia para todos y crear instituciones eficaces, responsables e inclusivas a todos los niveles} \cite{ODS16}. Según lo expuesto en los dos anteriores párrafos, se espera que este trabajo sirva para mejorar la eficacia de las instituciones públicas, en este caso de aquellas relacionadas con las ofertas de contrataciones.
    \\ \\
    
    \begin{figure}[h]
        \centering
        \includegraphics[width=0.625\textwidth]{ods16.jpeg}
        \captionof{figure}{Logo del Objetivo de Desarrollo Sostenible nº 16}
    \end{figure}

\newpage