\section{Correspondencias entre CODICE y OCDS}

    Aquí vendría una pequeña descripción de la sección

    \vspace{0.3cm}
    
    \subsection{Datos generales del expediente}
    
    Aquí vendría una pequeña descripción de la subsección
    
        \subsubsection{Estado}
            \begin{itemize}
                \item \textbf{Elemento en el esquema CODICE}:
                    \tabto{7.6cm} \texttt{cac-place-ext:ContractFolderStatus/} \\
                    \tabto{7.6cm} \texttt{\textit{cbc-place-ext:ContractFolderStatusCode}}
                \item \textbf{Elemento en el esquema OCDS}:
                    \tabto{7.6cm} \texttt{\textit{tag}}
                \item \textbf{Comentarios}: La lista de posibles códigos del elemento en CODICE se encuentra en el siguiente 
                    \href{https://contrataciondelestado.es/codice/cl/2.04/SyndicationContractFolderStatusCode-2.04.gc}{enlace}.
                    La lista análoga de códigos en el esquema OCDS se puede consultar en el siguiente
                    \href{https://standard.open-contracting.org/latest/es/schema/codelists/#release-tag}{enlace}.
                    La correspondencia entre ambas listas de códigos se ha establecido de la siguiente manera:
                        \subitem - \texttt{PRE} (Anuncio previo) $\rightarrow$ \texttt{planning}
                        \subitem - \texttt{PUB} (En plazo) $\rightarrow$ \texttt{tender}
                        \subitem - \texttt{EV} (Pendiente de adjudicación) $\rightarrow$ \texttt{tender}
                        \subitem - \texttt{ADJ} (Adjudicada) $\rightarrow$ \texttt{award}
                        \subitem - \texttt{RES} (Resuelta) $\rightarrow$ \texttt{contract}
                        \subitem - \texttt{ANUL} (Anulada) $\rightarrow$ \texttt{awardCancellation}
            \end{itemize}
        
        \subsubsection{Número de expediente}
            \begin{itemize}
                \item \textbf{Elemento en el esquema CODICE}:
                    \tabto{7.6cm} \texttt{cac-place-ext:ContractFolderStatus/} \\
                    \tabto{7.6cm} \texttt{\textit{cbc:ContractFolderID}}
                \item \textbf{Elemento en el esquema OCDS}:
                    \tabto{7.6cm} \texttt{\textit{ocid}}
                \item \textbf{Comentarios}: En el esquema de OCDS, los procesos de contratación deben identificarse mediante un código unívoco que será idéntico en cualquier posterior entrega del mismo proceso de contratación. Para asegurar que dichos códigos no puedan colisionar, OCDS provee a los publicadores prefijos que concatenar a los identificadores internos para así asegurar la creación de identificadores globales únicos.
            \end{itemize}
        
        \subsubsection{Objeto del contrato}
            \begin{itemize}
                \item \textbf{Elemento en el esquema CODICE}:
                    \tabto{7.6cm} \texttt{cac-place-ext:ContractFolderStatus/} \\
                    \tabto{7.6cm} \texttt{cac:ProcurementProject/} \\
                    \tabto{7.6cm} \texttt{\textit{cbc:Name}}
                \item \textbf{Elemento en el esquema OCDS}:
                    \tabto{7.6cm} \texttt{tender.\textit{title}}
            \end{itemize}
        
        \subsubsection{Valor estimado e importe de licitación}
            \begin{itemize}
                \item \textbf{Elementos en el esquema CODICE}:
                    \tabto{7.7cm} \texttt{cac-place-ext:ContractFolderStatus/} \\
                    \tabto{7.7cm} \texttt{cac:ProcurementProject/} \\
                    \tabto{7.7cm} \texttt{cac:BudgetAmount/} \\
                    \tabto{7.7cm} \{\texttt{\textit{cbc:EstimatedOverallContractAmount}}, \\
                    \tabto{7.7cm} \texttt{\textit{cbc:TotalAmount}}\}
                \item \textbf{Elementos en el esquema OCDS}:
                    \tabto{7.7cm} \texttt{EstimatedOverallContractAmount} \\ \tabto{8cm} $\rightarrow$ \texttt{tender.\textit{value}} \\
                    \tabto{7.7cm} \texttt{TotalAmount} \\ \tabto{8cm} $\rightarrow$ \texttt{planning.budget.\textit{amount}}
            \end{itemize}
        
        \subsubsection{Duración del contrato}
            \begin{itemize}
                \item \textbf{Elementos en el esquema CODICE}:
                    \tabto{7.7cm} \texttt{cac-place-ext:ContractFolderStatus/} \\
                    \tabto{7.7cm} \texttt{cac:ProcurementProject/} \\
                    \tabto{7.7cm} \texttt{cac:PlannedPeriod/} \\
                    \tabto{7.7cm} \{\texttt{\textit{cbc:StartDate}}, \\
                    \tabto{7.7cm} \{\texttt{\textit{cbc:EndDate}}, \\
                    \tabto{7.7cm} \texttt{\textit{cbc:DurationMeasure}}\}\}
                \item \textbf{Elementos en el esquema OCDS}:
                    \tabto{7.7cm} \texttt{StartDate} \\ \tabto{8cm} $\rightarrow$ \texttt{tender.tenderPeriod.\textit{startDate}} \\
                    \tabto{7.7cm} \texttt{EndDate} \\ \tabto{8cm} $\rightarrow$ \texttt{tender.tenderPeriod.\textit{endDate}}
                    \tabto{7.7cm} \texttt{DurationMeasure} \\ \tabto{8cm} $\rightarrow$ \texttt{tender.tenderPeriod.\textit{durationInDays}}
                \item \textbf{Comentarios}: La duración prevista del contrato puede expresarse mediante una fecha de inicio y una de final, o bien mediante una fecha de inicio y una duración prevista, cuyos valores pueden ser expresados mediante días (\texttt{DAY}), meses (\texttt{MON}) o años (\texttt{ANN}).
            \end{itemize}
        
        \subsubsection{Tipo de contrato}
            \begin{itemize}
                \item \textbf{Elemento en el esquema CODICE}:
                    \tabto{7.6cm} \texttt{cac-place-ext:ContractFolderStatus/} \\
                    \tabto{7.6cm} \texttt{cac:ProcurementProject/} \\
                    \tabto{7.6cm} \texttt{\textit{cbc:TypeCode}}
                \item \textbf{Elemento en el esquema OCDS}:
                    \tabto{7.6cm} \texttt{tender.\textit{mainProcurementCategory}}
                \item \textbf{Comentarios}: La lista de posibles valores que pueden tomar los códigos que describen los tipos de contrato en CODICE pueden encontrarse en el siguiente \href{https://contrataciondelestado.es/codice/cl/2.08/ContractCode-2.08.gc}{enlace}. La lista análoga de códigos en el esquema OCDS se puede consultar en el siguiente \href{https://standard.open-contracting.org/latest/es/schema/codelists/#procurement-category}{enlace}. La correspondencia entre ambas listas de códigos se ha establecido de la siguiente manera:
                        \subitem - \texttt{1} (Suministros) $\rightarrow$ \texttt{goods}
                        \subitem - \texttt{2} (Servicios) $\rightarrow$ \texttt{services}
                        \subitem - \texttt{3} (Obras) $\rightarrow$ \texttt{works}
                        \subitem - \texttt{21} (Gestión de servicios públicos) $\rightarrow$ \texttt{services}
                        \subitem - \texttt{22} (Gestión de servicios) $\rightarrow$ \texttt{services}
                        \subitem - \texttt{31} (Concesión de obras públicas) $\rightarrow$ \texttt{works}
                        \subitem - \texttt{32} (Concesión de obras) $\rightarrow$ \texttt{works}
            \end{itemize}
        
        \subsubsection{Enlaces de descarga de pliegos}
            \begin{itemize}
                \item \textbf{Elementos en el esquema CODICE}:
                    \tabto{7.7cm} \texttt{cac-place-ext:ContractFolderStatus/} \\
                    \tabto{7.7cm} \{\texttt{cac:LegalDocumentReference/}, \\
                    \tabto{7.7cm} \texttt{cac:TechnicalDocumentReference/}, \\
                    \tabto{7.7cm} \texttt{cac:AditionalDocumentReference/}\} \\
                    \tabto{7.7cm} \texttt{\textit{cbc:ID}},\\
                    \tabto{7.7cm} \texttt{cac:Attachment/} \\
                    \tabto{7.7cm} \texttt{cac:ExternalReference/} \\
                    \tabto{7.7cm} \texttt{\textit{cbc:URI}}
                \item \textbf{Elemento en el esquema OCDS}:
                    \tabto{7.7cm} \texttt{tender.\textit{documents}}
                \item \textbf{Comentarios}: Los distintos pliegos que pueden documentar un proceso de contratación vienen descritos, en su tipo, por el elemento inmediatamente inferior a \texttt{ContractFolderStatus}, y su mapeo al esquema OCDS se representa mediante la lista de códigos \href{https://standard.open-contracting.org/latest/es/schema/codelists/#document-type}{\texttt{documentType}}.
                        \subitem - LegalDocumentReference $\rightarrow$ \texttt{?}
                        \subitem - TechnicalDocumentReference $\rightarrow$ \texttt{technicalSpecifications}
                        \subitem - AditionalDocumentReference $\rightarrow$ \texttt{?}
            \end{itemize}

    \vspace{0.3cm}
    
    \subsection{Lotes}
        
        El uso de lotes en el esquema OCDS viene otorgado por  \href{https://extensions.open-contracting.org/en/extensions/lots/v1.1.5/}{la extensión de mismo nombre}.
    
        \subsubsection{Número de lote}
            \begin{itemize}
                \item \textbf{Elemento en el esquema CODICE}:
                    \tabto{7.6cm} \texttt{cac-place-ext:ContractFolderStatus/} \\
                    \tabto{7.6cm} \texttt{cac:ProcurementProjectLot/} \\
                    \tabto{7.6cm} \texttt{\textit{cbc:ID}}
                \item \textbf{Elementos en el esquema OCDS}:
                    \tabto{7.6cm} \texttt{tender.lots[i].\textit{id}} \\
                    \tabto{7.6cm} \texttt{tender.items[i].\textit{id}} \\
                    \tabto{7.6cm} \texttt{tender.items[i].\textit{relatedLot}} \\
                \item \textbf{Comentarios}: En todos los apartados de esta sección se utilizará la notación \texttt{lots[i]} e \texttt{items[i]} para referirse a los i-ésimos objetos representando el lote y el artículo, respectivamente, dentro de las colecciones de éstos. El campo \texttt{relatedLot} se utilizará para enlazar el artículo con su lote.
            \end{itemize}
            
        \subsubsection{Objeto del lote}
            \begin{itemize}
                \item \textbf{Elemento en el esquema CODICE}:
                    \tabto{7.6cm} \texttt{cac-place-ext:ContractFolderStatus/} \\
                    \tabto{7.6cm} \texttt{cac:ProcurementProjectLot/} \\
                    \tabto{7.6cm} \texttt{cac:ProcurementProject/} \\
                    \tabto{7.6cm} \texttt{\textit{cbc:Name}}
                \item \textbf{Elemento en el esquema OCDS}:
                    \tabto{7.6cm} \texttt{tender.lots[i].\textit{name}}
            \end{itemize}
        
        \subsubsection{Importe del lote}
            \begin{itemize}
                \item \textbf{Elemento en el esquema CODICE}:
                    \tabto{7.6cm} \texttt{cac-place-ext:ContractFolderStatus/} \\
                    \tabto{7.6cm} \texttt{cac:ProcurementProjectLot/} \\
                    \tabto{7.6cm} \texttt{cac:ProcurementProject/} \\
                    \tabto{7.6cm} \texttt{cac:BudgetAmount/} \\
                    \tabto{7.6cm} \texttt{\textit{cbc:TotalAmount}}
                \item \textbf{Elemento en el esquema OCDS}:
                    \tabto{7.6cm} \texttt{tender.lots[i].\textit{value}}
            \end{itemize}
            
        \subsubsection{Clasificación \textit{CPV}}
            \begin{itemize}
                \item \textbf{Elemento en el esquema CODICE}:
                    \tabto{7.6cm} \texttt{cac-place-ext:ContractFolderStatus/} \\
                    \tabto{7.6cm} \texttt{cac:ProcurementProjectLot/} \\
                    \tabto{7.6cm} \texttt{cac:ProcurementProject/} \\
                    \tabto{7.6cm} \texttt{cac:RequiredCommodityClassification/} \\
                    \tabto{7.6cm} \texttt{\textit{cbc:ItemClassificationCode}}
                \item \textbf{Elemento en el esquema OCDS}:
                    \tabto{7.6cm} \texttt{tender.lots[i].\textit{classification}}
                \item \textbf{Comentarios}: El sistema de clasificación de bienes y servicios utilizado en CODICE, \href{https://www.licitaciones.es/codigos-cpv}{\textit{CPV}}, es un estándar válido en OCDS, indicándolo en el campo \texttt{classification.\textit{schema}}.
            \end{itemize}

    \vspace{0.3cm}

    \subsection{Procesos de licitación}
    
        Aquí vendría una pequeña descripción de la subsección
    
        \subsubsection{Tipo de procedimiento}
            \begin{itemize}
                \item \textbf{Elemento en el esquema CODICE}:
                    \tabto{7.6cm} \texttt{cac-place-ext:ContractFolderStatus/} \\
                    \tabto{7.6cm} \texttt{cac:TenderingProcess/} \\
                    \tabto{7.6cm} \texttt{\textit{cbc:ProcedureCode}}
                \item \textbf{Elemento en el esquema OCDS}:
                    \tabto{7.6cm} \texttt{tender.\textit{procurementMethod}}
                \item \textbf{Comentarios}: La lista de posibles códigos del elemento en CODICE se encuentra en el siguiente 
                    \href{https://contrataciondelestado.es/codice/cl/2.07/SyndicationTenderingProcessCode-2.07.gc}{enlace}.
                    La lista análoga de códigos en el esquema OCDS se puede consultar en el siguiente
                    \href{https://standard.open-contracting.org/latest/es/schema/codelists/#method}{enlace}.
                    La correspondencia entre ambas listas de códigos se ha establecido de la siguiente manera:
                        \subitem - \texttt{1} (Abierto) $\rightarrow$ \texttt{open}
                        \subitem - \texttt{2} (Restringido) $\rightarrow$ \texttt{selective}
                        \subitem - \texttt{3} (Negociado sin publicidad) $\rightarrow$ \texttt{limited}
                        \subitem - \texttt{4} (Negociado con publicidad) $\rightarrow$ \texttt{limited}
                        \subitem - \texttt{5} (Diálogo competitivo) $\rightarrow$ \texttt{open}
                        \subitem - \texttt{6} (Contrato menor) $\rightarrow$ \texttt{open}
                        \subitem - \texttt{7} (Derivado de acuerdo macro) $\rightarrow$ \texttt{selective}
                        \subitem - \texttt{8} (Concurso de proyectos) $\rightarrow$ \texttt{open}
                        \subitem - \texttt{9} (Abierto simplificado) $\rightarrow$ \texttt{open}
                        \subitem - \texttt{10} (Asociación para la innovación) $\rightarrow$ \texttt{limited}
                        \subitem - \texttt{11} (Derivado de asociación para la innovación) $\rightarrow$ \texttt{limited}
                        \subitem - \texttt{12} (Basado en un sistema dinámico de adquisición) $\rightarrow$ \texttt{open}
                        \subitem - \texttt{13} (Licitación con negociación) $\rightarrow$ \texttt{open}
                        \subitem - \texttt{100} (Normas internas) $\rightarrow$ \texttt{limited}
            \end{itemize}
        
        \subsubsection{Sistema de contratación} \label{subsec:SistemaDeContratacion}
            \begin{itemize}
                \item \textbf{Elemento en el esquema CODICE}:
                    \tabto{7.6cm} \texttt{cac-place-ext:ContractFolderStatus/} \\
                    \tabto{7.6cm} \texttt{cac:TenderingProcess/} \\
                    \tabto{7.6cm} \texttt{\textit{cbc:ContractingSystemCode}}
                \item \textbf{Elemento en el esquema OCDS}:
                    \tabto{7.6cm} \texttt{tender.\textit{procurementMethodDetails\_es}}
                \item \textbf{Comentarios}: Este elemento describe si el sistema de contratación se trata de un contrato, de un acuerdo marco, o de un sistema dinámico de adquisición. Con el sufijo \texttt{es} se especifica que el campo muestra su información en castellano, tal y como aparece en el \href{https://contrataciondelestado.es/codice/cl/2.08/ContractingSystemTypeCode-2.08.gc}{documento de códigos} de CODICE.
            \end{itemize}
        
        \subsubsection{Presentación de la oferta} \label{subsec:PresentacionOferta}
            \begin{itemize}
                \item \textbf{Elemento en el esquema CODICE}:
                    \tabto{7.6cm} \texttt{cac-place-ext:ContractFolderStatus/} \\
                    \tabto{7.6cm} \texttt{cac:TenderingProcess/} \\
                    \tabto{7.6cm} \texttt{\textit{cbc:SubmissionMethodCode}}
                \item \textbf{Elemento en el esquema OCDS}:
                    \tabto{7.6cm} \texttt{tender.\textit{submissionMethod}}
                \item \textbf{Comentarios}: La lista de posibles códigos del elemento en CODICE se encuentra en el siguiente 
                    \href{https://contrataciondelestado.es/codice/cl/1.04/TenderDeliveryCode-1.04.gc}{enlace}.
                    La lista análoga de códigos en el esquema OCDS se puede consultar en el siguiente
                    \href{https://standard.open-contracting.org/latest/es/schema/codelists/#submission-method}{enlace}.
                    La correspondencia entre ambas listas de códigos se ha establecido de la siguiente manera:
                        \subitem - \texttt{1} (Electrónica) $\rightarrow$ \texttt{[electronicSubmission]}
                        \subitem - \texttt{2} (Manual) $\rightarrow$ \texttt{[written]}
                        \subitem - \texttt{3} (Manual y/o Electrónica) $\rightarrow$ \texttt{[electronicSubmission, written]}
            \end{itemize}
        
        \subsubsection{Idioma de presentación de la oferta}
            \begin{itemize}
                \item \textbf{Elemento en el esquema CODICE}:
                    \tabto{7.6cm} \texttt{cac-place-ext:ContractFolderStatus/} \\
                    \tabto{7.6cm} \texttt{cac:TenderingTerms/} \\
                    \tabto{7.6cm} \texttt{cac:Language/} \\
                    \tabto{7.6cm} \texttt{\textit{cbc:ID}}
                \item \textbf{Elemento en el esquema OCDS}:
                    \tabto{7.6cm} \texttt{tender.\textit{submissionMethodDetails}}
            \end{itemize}
        
        \subsubsection{Licitación son subasta electrónica}
            \begin{itemize}
                \item \textbf{Elemento en el esquema CODICE}:
                    \tabto{7.6cm} \texttt{cac-place-ext:ContractFolderStatus/} \\
                    \tabto{7.6cm} \texttt{cac:TenderingProcess/} \\
                    \tabto{7.6cm} \texttt{cac:AuctionTerms/} \\
                    \tabto{7.6cm} \texttt{\textit{cbc:AuctionConstraintIndicator}}
                \item \textbf{Elemento en el esquema OCDS}:
                    \tabto{7.6cm} \texttt{tender.\textit{submissionMethod}}
                \item \textbf{Comentarios}: Si en el elemento \texttt{AuctionConstraintIndicator} está indicado el valor \texttt{true}, significa que la licitación estará sujeta a subasta electrónica, por lo que el valor \texttt{electronicAuction} de la lista de códigos \href{https://standard.open-contracting.org/latest/es/schema/codelists/#submission-method}{\texttt{submissionMethod}} se sobreescribe en el campo \texttt{tender.submissionMethod}, que ya había sido inicializado por el elemento \hyperref[subsec:PresentacionOferta]{Presentación de la oferta}.
            \end{itemize}
        
    \vspace{0.3cm}
    
    \subsection{Entidades adjudicadoras}
    
        Aquí vendría una pequeña descripción de la subsección
    
        \subsubsection{Órgano de contratación}
            \begin{itemize}
                \item \textbf{Elemento en el esquema CODICE}:
                    \tabto{7.6cm} \texttt{cac-place-ext:ContractFolderStatus/} \\
                    \tabto{7.6cm} \texttt{cac-place-ext:LocatedContractingParty/} \\
                    \tabto{7.6cm} \texttt{cac:Party/} \\
                    \tabto{7.6cm} \texttt{cac:PartyName/} \\
                    \tabto{7.6cm} \texttt{\textit{cbc:Name}}
                \item \textbf{Elemento en el esquema OCDS}:
                    \tabto{7.6cm} \texttt{parties[i].\textit{name}}
                \item \textbf{Comentarios}: En todos los apartados de esta sección se utilizará la notación \texttt{parties[i]} para referirse al mismo objeto representando a la entidad adjudicadora, debido a que \texttt{parties} es una colección de objetos representando a todas las partes involucradas.
            \end{itemize}
        
        \subsubsection{Ubicación orgánica} \label{subsec:UbicacionOrganica}
            \begin{itemize}
                \item \textbf{Elemento en el esquema CODICE}:
                    \tabto{7.6cm} \texttt{cac-place-ext:ContractFolderStatus/} \\
                    \tabto{7.6cm} \texttt{cac-place-ext:LocatedContractingParty/} \\
                    \tabto{7.6cm} \texttt{cac:Party/} \\
                    \tabto{7.6cm} \texttt{cac:PartyIdentification/} \\
                    \tabto{7.6cm} \texttt{\textit{cbc:ID}}
                \item \textbf{Elementos en el esquema OCDS}:
                    \tabto{7.6cm} \texttt{parties[i].\textit{id}} \\
                    \tabto{7.6cm} \texttt{parties[i].\textit{identifier.schema}} \\
                    \tabto{7.6cm} \texttt{parties[i].\textit{identifier.id}} \\
                    \tabto{7.6cm} \texttt{parties[i].\textit{additionalIdentifiers}} \\
                    \tabto{7.6cm} \texttt{parties[i].\textit{roles}}
                \item \textbf{Comentarios}: Siguiendo la directriz del estándar OCDS sobre los \href{https://standard.open-contracting.org/latest/es/schema/codelists/#organization-identifier-scheme}{esquemas de identificación de organizaciones}, estos elementos se procesan dependiendo de su esquema: \href{http://org-id.guide/list/ES-DIR3}{DIR3} o NIF (\href{http://org-id.guide/list/ES-RMC}{RMC}). Adicionalmente, si \texttt{PartyIdentification} contiene ambos identificadores, se utilizará el campo \texttt{additionalIdentifiers} para mapear la máxima cantidad de información. Por último, en este elemento de mapeo se ha añadido el rol de la parte adjudicadora (\texttt{procuringEntity}).
            \end{itemize}
        
        \subsubsection{Otros campos}
            \begin{itemize}
                \item \textbf{Elementos en el esquema CODICE}:
                    \tabto{7.6cm} \texttt{cac-place-ext:ContractFolderStatus/} \\
                    \tabto{7.6cm} \texttt{cac-place-ext:LocatedContractingParty/} \\
                    \tabto{7.6cm} \texttt{cac:Party/} \\
                    \tabto{7.6cm} \texttt{\textit{cac:PostalAddress}} \\
                    \tabto{7.6cm} \texttt{\textit{cac:Contact}} \\
                \item \textbf{Elementos en el esquema OCDS}:
                    \tabto{7.6cm} \texttt{parties[i].\textit{address}} \\
                    \tabto{7.6cm} \texttt{parties[i].\textit{countryName}} \\
                    \tabto{7.6cm} \texttt{parties[i].\textit{contactPoint}}
            \end{itemize}
    
    \vspace{0.3cm}
    
    \subsection{Resultado del procedimiento}
        
        Aquí vendría una pequeña descripción de la subsección
    
        \subsubsection{Identificador}
            \begin{itemize}
                \item \textbf{Elemento en el esquema CODICE}:
                    \tabto{7.6cm} \texttt{cac-place-ext:ContractFolderStatus/} \\
                    \tabto{7.6cm} \texttt{cac:TenderResult/} \\
                    \tabto{7.6cm} \texttt{cac:AwardedTenderedProject/} \\
                    \tabto{7.6cm} \texttt{\textit{cbc:ProcurementProjectLotID}}
                \item \textbf{Elemento en el esquema OCDS}:
                    \tabto{7.6cm} \texttt{awards[i].\textit{id}}
                \item \textbf{Comentarios}: En todos los apartados de esta sección se utilizará la notación \texttt{awards[i]} para referirse al mismo objeto representando el resultado de un procedimiento de adjudicación dentro de un mismo proceso de contratación.
            \end{itemize}
            
        \subsubsection{Resultado}
            \begin{itemize}
                \item \textbf{Elemento en el esquema CODICE}:
                    \tabto{7.6cm} \texttt{cac-place-ext:ContractFolderStatus/} \\
                    \tabto{7.6cm} \texttt{cac:TenderResult/} \\
                    \tabto{7.6cm} \texttt{\textit{cbc:ResultCode}}
                \item \textbf{Elemento en el esquema OCDS}:
                    \tabto{7.6cm} \texttt{awards[i].\textit{status}}
                \item \textbf{Comentarios}: La lista de posibles códigos del elemento en CODICE se encuentra en el siguiente 
                    \href{http://contrataciondelestado.es/codice/cl/2.02/TenderResultCode-2.02.gc}{enlace}.
                    La lista análoga de códigos en el esquema OCDS se puede consultar en el siguiente
                    \href{https://standard.open-contracting.org/latest/en/schema/codelists/#award-status}{enlace}.
                    La correspondencia entre ambas listas de códigos se ha establecido de la siguiente manera:
                        \subitem - \texttt{1} (Adjudicado provisionalmente) $\rightarrow$ \texttt{pending}
                        \subitem - \texttt{2} (Adjudicado definitivamente) $\rightarrow$ \texttt{active}
                        \subitem - \texttt{3} (Desierto) $\rightarrow$ \texttt{cancelled}
                        \subitem - \texttt{4} (Desistimiento) $\rightarrow$ \texttt{unsuccessful}
                        \subitem - \texttt{5} (Renuncia) $\rightarrow$ \texttt{unsuccessful}
                        \subitem - \texttt{6} (Desierto provisionalmente) $\rightarrow$ \texttt{pending}
                        \subitem - \texttt{7} (Desierto definitivamente) $\rightarrow$ \texttt{cancelled}
                        \subitem - \texttt{8} (Adjudicado) $\rightarrow$ \texttt{active}
                        \subitem - \texttt{9} (Formalizado) $\rightarrow$ \texttt{active}
                        \subitem - \texttt{10} (Licitador mejor valorado: requerimiento de documentación) $\rightarrow$ \texttt{active}
            \end{itemize}
            
        \subsubsection{Identidad del adjudicatario}
            \begin{itemize}
                \item \textbf{Elementos en el esquema CODICE}:
                    \tabto{7.6cm} \texttt{cac-place-ext:ContractFolderStatus/} \\
                    \tabto{7.6cm} \texttt{cac:TenderResult/} \\
                    \tabto{7.6cm} \texttt{cac:WinningParty/} \\
                    \tabto{7.7cm} \{\texttt{\textit{cac:PartyIdentification}}, \\
                    \tabto{7.7cm} \texttt{\textit{cac:PartyName}}\}
                \item \textbf{Elementos en el esquema OCDS}:
                    \tabto{7.6cm} \texttt{awards[i].\textit{suppliers}} \\
                    \tabto{7.6cm} \texttt{\textit{parties[j]}}
                \item \textbf{Comentarios}: Debido a que este mapeo introduce en el documento la identidad del adjudicatario, también debe introducir en el campo \texttt{parties} la referencia a dicha parte involucrada. De manera análoga al mapeo que realiza  \hyperref[subsec:UbicacionOrganica]{Ubicación orgánica}, el campo \texttt{identifier} tanto en el campo \texttt{suppliers} como en la colección \texttt{parties} detecta si la identificación de la entidad es del tipo DIR3 o NIF, indicándolo en el campo \texttt{schema}. Por último, este mapeo también introduce en la referencia al adjudicatario en \texttt{parties} en el campo \texttt{roles} el valor \texttt{supplier}, tal y como se indica en la lista de códigos de OCDS sobre \href{https://standard.open-contracting.org/latest/en/schema/codelists/#party-role}{roles de las partes involucradas}.
            \end{itemize}
            
        \subsubsection{Importe de adjudicación}
            \begin{itemize}
                \item \textbf{Elemento en el esquema CODICE}:
                    \tabto{7.6cm} \texttt{cac-place-ext:ContractFolderStatus/} \\
                    \tabto{7.6cm} \texttt{cac:TenderResult/} \\
                    \tabto{7.6cm} \texttt{cac:AwardedTenderedProject/} \\
                    \tabto{7.6cm} \texttt{cac:LegalMonetaryTotal/} \\
                    \tabto{7.6cm} \texttt{\textit{cbc:PayableAmount}}
                \item \textbf{Elemento en el esquema OCDS}:
                    \tabto{7.6cm} \texttt{awards[i].\textit{value}}
            \end{itemize}
            
        \subsubsection{Número de licitadores participantes}
            \begin{itemize}
                \item \textbf{Elemento en el esquema CODICE}:
                    \tabto{7.6cm} \texttt{cac-place-ext:ContractFolderStatus/} \\
                    \tabto{7.6cm} \texttt{cac:TenderResult/} \\
                    \tabto{7.6cm} \texttt{\textit{cbc:ReceivedTenderQuantity}}
                \item \textbf{Elemento en el esquema OCDS}:
                    \tabto{7.6cm} \texttt{tender.\textit{numberOfTenderers}}
                \item \textbf{Comentarios}: Dado que OCDS no provee el campo \texttt{numberOfTenderers} dentro de la colección de \texttt{awards}, este mapeo sólo se realizará en el caso de que el proceso de adjudicación sólo contenga un lote, es decir, la longitud de la colección \texttt{awards} sea 1 y, por tanto, \texttt{numberOfTenderers} refleje efectivamente el número de participantes.
            \end{itemize}
            
        \subsubsection{Fecha de la adjudicación}
            \begin{itemize}
                \item \textbf{Elemento en el esquema CODICE}:
                    \tabto{7.6cm} \texttt{cac-place-ext:ContractFolderStatus/} \\
                    \tabto{7.6cm} \texttt{cac:TenderResult/} \\
                    \tabto{7.6cm} \texttt{\textit{cbc:AwardDate}}
                \item \textbf{Elementos en el esquema OCDS}:
                    \tabto{7.6cm} \texttt{awards[i].\textit{date}}
            \end{itemize}
            
        \subsubsection{Descripción de la adjudicación}
            \begin{itemize}
                \item \textbf{Elemento en el esquema CODICE}:
                    \tabto{7.6cm} \texttt{cac-place-ext:ContractFolderStatus/} \\
                    \tabto{7.6cm} \texttt{cac:TenderResult/} \\
                    \tabto{7.6cm} \texttt{\textit{cbc:Description}}
                \item \textbf{Elemento en el esquema OCDS}:
                    \tabto{7.6cm} \texttt{awards[i].\textit{description\_es}}
                \item \textbf{Comentarios}: Como sucedía en el elemento  \hyperref[subsec:SistemaDeContratacion]{sistema de contratación}, se hace uso del sufijo \texttt{es} para indicar que el campo contendrá su información en castellano.
            \end{itemize}
            
\newpage