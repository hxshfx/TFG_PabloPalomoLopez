\mysection{Introducción y objetivos}

    La contratación pública es una actividad que abarca la práctica totalidad de sectores laborales y organizaciones públicas y privadas. En cada uno de los años entre 2015 y 2017, el mercado de licitaciones de trabajos, bienes y servicios europeo tuvo movimientos anuales de más de 2 trillones de euros, lo que representa más del 13\% del producto interior bruto (PIB) de la Unión Europea \cite{UEGDP1} \cite{UEGDP2}.
    \\ \\
    Sin embargo, tanto gobiernos como compañías de participación estatal se ven envueltos en desafíos de gran dimensión: deben proveer servicios con presupuestos lo más reducidos posibles, prevenir las pérdidas económicas y de confianza que generan las actividades fraudulentas como la corrupción, a la vez que tratan de construir saludables y vigorosos aparatos económicos. Además, en coyunturas como la actual en las que la actividad económica se ve gravemente disminuida, estos desafíos son aún mayores. Por ello, la necesidad de disponer de herramientas útiles para el acceso y análisis de los datos de contrataciones públicas es más acuciante que nunca: la ausencia de dichas utilidades en un ámbito tan complejo como éste sólo puede desembocar en una peor toma de decisiones por parte de las entidades adjudicadoras, lo cual repercute negativamente en las posibles entidades adjudicatarias.
    \\ \\
    Teniendo en cuenta los elementos que motivan la consecución de este trabajo, se han explorado tanto las herramientas disponibles actualmente en el ámbito de la contratación pública, como las guías de referencia, o estándares, que rigen la publicación de datos tanto en el plano nacional como en el europeo. Con respecto a los proyectos que en la actualidad se encuentran involucrados en el desarrollo de servicios útiles este campo, destaca \textit{TheyBuyForYou} \cite{TBFY}, un equipo que aglomera todo tipo de organizaciones, empresas, universidades, entidades gubernamentales, etc., cuyo objetivo es la construcción de diversas tecnologías, en el ámbito \textit{web} y mediante \textit{APIs}, que permitan la publicación, integración, análisis y visualización de un masivo grafo de conocimiento capaz de recabar información de gasto público y sus entidades involucradas a través de múltiples fuentes de datos a lo largo de la Unión Europea \cite{TBFYPAPER}.
    \\ \\
    Este trabajo de recopilación de datos bajo unas mismas reglas que los describan sólo es posible mediante el uso de estándares; que rijan la forma en la que los datos se estructuran, los formatos que utilizan las entidades que publican estos datos, los lenguajes en los que esos datos se encuentran expresados, etc. Es por ello que estándares como \texttt{CODICE}, el utilizado por las organizaciones gubernamentales españolas, necesitan adecuarse a los estándares europeos (concretamente, \texttt{OCDS}), dado que son éstos los utilizados por los proyectos como \textit{TheyBuyForYou}, que tratan de integrar la mayor cantidad posible de datos de contrataciones públicas de los países comunitarios.
    \\ \\
    Debido a que no existe ningún marco capaz de preservar la expresividad de los datos bajo el esquema \texttt{CODICE} a \texttt{OCDS}, lo que pretende este trabajo es construir tanto el sistema como las reglas capaces de realizar esta transformación con el objetivo de producir datos útiles que permitan tanto su análisis y exploración, como la integración con los datos de otros países en grafos de conocimiento como los de \textit{TheyBuyForYou}.
\newpage